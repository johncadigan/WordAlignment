\documentclass[15pt]{article}
\usepackage{graphicx}
\title{Ling 575: Alignment}
\author{John Cadigan}
\begin{document} 
\subsection*{Ling 575: Alignment}
John Cadigan\\
github: https://github.com/johncadigan/WordAlignment.git \\
run: compile2.10.sh then run.sh \\

\textbf{Motivation}\\
I explored some multi-threaded and combined alignment approaches to improve the model$,$ but I did not have much success. I did find ways of formulating the calculations which will resemble those I will use on Spark for my project.

\textbf{Description}\\
\textit{IBM Model 1}\\
I used the standard IBM Model 1. To help it converge faster$,$ I set every target word e and is set to to be equally probable from each of the n words from f it is found with
$P(e|f) = \frac{1}{n+1}$ \\
With each iteration of the EM algorithm$,$ two maps are used to store the expected probability counts are created: $count(e|f)$ and $count(f)$. Then for each sentence the deltas are calculated for every target word; the delta is the sum of all the translation probabilities which apply to target word \\
$delta(e_i) = \sum_{f \epsilon F}P(e_i|f)$ \\
For each possible combination of e and f$,$ the counts are both incremented by the probability divided by delta \\
$count(e|f)+= \frac{P(e|f)}{delta(e)}$
$count(f)+= \frac{P(e|f)}{delta(e)}$ \\
After all sentences$,$ these counts are normalized to produce newer, more accurate estimations of $P(e|f)$ and the process continues until all iterations are complete \\
$P(e|f)=\frac{count(e|f)}{count(f)}$ \\
\textit{Alignment and symmetrization}\\
I then go through every sentence and output the highest probable alignment for each word based only on $P(e|f)$. I then unify the output from models trained in each direction to recover more translations and only report translations found by both.  \\
\textbf{Results}\\
    \begin{tabular}{| l| l | l | l |}
    \hline
     Iterations & Precision & Recall & AER \\ \hline
    5 & 0.870 & 0.659 & .235 \\ \hline
	10 & 0.866 &  0.671 & 0.229 \\ \hline
    20 & 0.863 &  0.674 & 0.229 \\ \hline
    \end{tabular} \\
\\  
\end{document}